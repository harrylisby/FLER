\documentclass[12pt,letterpaper]{IEEEtran}
\usepackage[utf8]{inputenc}
\usepackage[spanish,es-tabla]{babel}
\usepackage{mathtools}
\usepackage{enumitem}
\usepackage{graphicx}
\usepackage{multirow}
\usepackage{float}
\usepackage{listings}
\usepackage{color}
\usepackage{graphicx}
\graphicspath{{Imagenes/}}

\definecolor{dkgreen}{rgb}{0,0.6,0}
\definecolor{gray}{rgb}{0.5,0.5,0.5}
\definecolor{mauve}{rgb}{0.58,0,0.82}

\lstset{frame=tb,
  language=VHDL,
  aboveskip=3mm,
  belowskip=3mm,
  showstringspaces=false,
  columns=flexible,
  basicstyle={\small\ttfamily},
  numbers=none,
  numberstyle=\tiny\color{gray},
  keywordstyle=\color{blue},
  commentstyle=\color{dkgreen},
  stringstyle=\color{mauve},
  breaklines=true,
  breakatwhitespace=true,
  tabsize=3
}

\title{Informe de Proyecto: FLER(Four Leg Explorer Robot) }
\author{Integrantes:Alison Lisby, Emmanuel Rodríguez,Kenneth Leiva, Harry Lisby}

\date {today}

\begin{document}
\maketitle 
\renewcommand{\leftmark}{UNIVERSIDAD LATINA DE COSTA RICA -- BINGE-61 Microcontroladores}

\begin{abstract}
Este proyecto tiene como objetivo la aplicación de los conceptos aprendidos de los dispositivos de open source, desde  los protocolos de conexión que tienen hasta su capacidades a la hora de controlar otros dispositivos.
		
\end{abstract}

\section{Descripción del Proyecto}
El FLER es un dispositivo capaz de desplazarse de manera autónoma evitando obstáculos, que además tendrá la capacidad de muestrear las condiciones ambientales de un área específica y compartirlas de manera remota con el usuario, con el fin de determinar si las condiciones son seguras para un humano.



 

\section{Tareas a realizar}
\begin{itemize}
  \item Instalar el sensor ultrasónico HC-SR04 en una placa arduino UNO y desarrollar el código respectivo para detectar obstáculos.
  \item Instalar un sensor infrarrojo con su respectivo código para detección de obstáculos.
  \item  Determinar cual es el mejor sensor en términos de sensibilidad y respuesta ante la presencia de objetos.
  \item Instalar el sensor DTH22 en una placa arduino UNO y desarrollar el código correspondiente para el monitoreo de temperatura y humedad relativa.
  \item Instalar el sensor MQ-2 en la misma placa y desarrollar el código respectivo para el monitoreo de gases peligrosos en el ambiente.
  \item  Analizar los datos y validar la exactitud de los mismos comparándolos contra un tercer elemento (medidor de condiciones ambientales).
  \item Investigar la mejor opción para el mapeo de los datos a través de internet por ejemplo ``pubnub'' ó ``Cayenne''.
  \item Investigar y realizar pruebas con el microcontrolador ES8266 NodeMCU para la transmisión de datos por internet.
  \item Desarrollar el código correspondiente para el control de los motores y unidad principal mediante un modelo cinemático, utilizando el microcontrolador STM32F103.
  \item Investigar el uso de acelerómetros para el control de estabilidad. 
  \item Eventualmente, instalar una cámara y configurarla através de una plataforma raspberry pi 3.
  \item Investigar y configurar la opción de video de la cámara de manera que se pueda acceder por internet.
  
   
 
   
\end{itemize}

\end{document}